\documentclass[12pt]{article}
% \usepackage{sbc-template}
\usepackage{graphicx,url}
%\usepackage[dvips]{graphicx}
%\usepackage[pdftex]{color,graphicx}
\usepackage{multicol}
\usepackage{verbatim}
\usepackage{array}
\usepackage{amssymb,amsmath}
\usepackage[brazilian]{babel}
\usepackage[utf8]{inputenc}
\usepackage[T1]{fontenc}
\usepackage{indentfirst}
\usepackage{listings}   
\usepackage{pdfpages}
\sloppy

\title{Fórmulas}

\begin{document}

\includepdf[pages=-, fitpaper=true]{parte1.pdf}

\includepdf[pages=-, fitpaper=true]{SphericalCap.pdf}

% parte1:
% \includepdf[pages=-, fitpaper=true]{formularioGeomEsp.pdf}


% \includepdf[pages=-, fitpaper=true]{rotacionar.pdf}

% parte1:
% \begin{figure}[!t]
% \centering
% \includegraphics[width=\textwidth]{formulas2d.png}
% \label{f2d}
% \end{figure}

% parte 1:
% \begin{figure}[!t]
% \centering
% \includegraphics[width=\textwidth]{formulas3d.png}
% \label{f3d}
% \end{figure}

\subsection*{Fórmulas da Elipse}

Equações paramétricas: $$\begin{cases} x(t) = a \cos t \\ y(t) = b \sin t \end{cases}$$

Equação cartesiana: $$\dfrac{x^2}{a^2} + \dfrac{y^2}{b^2} = 1 $$

Equação polar: $$ r(\theta) = \dfrac{ab}{\sqrt{(b \cos \theta)^2 + (a \sin \theta)^2 }} $$

Área: $$ A = ab\pi $$ 

Comprimento do arco: $$ s = 4a \cdot E\Big(1 - \dfrac{a^2}{b^2}\Big)$$

O parâmetro focal da elipse é: $$p = \dfrac{b^2}{\sqrt{a^2-b^2}} = \dfrac{a^2-c^2}{c} = \dfrac{a(1-e^2)}{e}$$

\subsection*{Pick Theorem}

Só serve para polígonos com coordenadas inteiras: $A = i + \dfrac{b}{2} - 1$, onde:

\begin{itemize}
    \item $A = $ Área do polígono
    \item $i = $ Número de pontos com coordenadas inteiras dentro do polígono
    \item $b = $ número de pontos com coordenadas inteiras no perímetro do polígono
\end{itemize}

\subsection*{Probabilidade}

Fórumulas de probabilidade condicional:

$$P(A | B) = P(A \cap B) / P(B)$$
$$P(A \cap B) = P(A|B) \cdot P(B)$$
$$P(A \cup B) = P(A) + P(B) - P(A \cap B)$$

\subsection*{Fórmula de Stirling}

$$ n! \approx \sqrt{2\pi n} \Big(\frac{n}{e}\Big)^n $$

\subsection*{Ternos Pitagóricos}

Para quaisquer $a, b \in \mathbb{N}$, são ternos pitagóricos:

$$ \begin{cases}x = a^2 - b^2 \\ y = 2ab \\ z = a^2 + b^2\end{cases} $$

\subsection*{Equações diofantinas}

A equação diofantina linear $ax + by = n$ tem solução inteira, se e somente se, $n | mdc(a, b)$.

A solução da equação pode ser obtida pelo algoritmo de euclides extendido, obtendo os coeficientes $X$ e $Y$ tal que:

$$aX + bY = 1 \Rightarrow aXn + bYn = n $$

Daí, derivam-se as soluções gerais: 
$$\begin{cases}x = Xn + bt \\ y = Yn - at \end{cases}$$


\subsection*{Critérios de divisibilidade bizarros}

\begin{itemize}
\item Divisibilidade geral: Pra saber se um número $n$ é divísível por $k$, verificar se $n$ é divisível pelas maiores potências da decomposição de $k$ em fatores primos.

\item Divisibilidade por $2^k$: Um número é divisível por $2^k$ se o número formado pelos últimos k algarismos for divisível $2^k$.

\item Dvisibilidade por $7$: Um número é divisível por $7$ se o dobro do último algarismo, subtraído do número sem o último algarismo, for divisível por $7$. 

\item Divisibilidade por $11$: Um número é divisível por $11$ se a soma dos algarismos de ordem par $S_p$ menos a soma dos algarismos de ordem ímpar $S_i$ for um número divisível por $11$. Obs: o algarismo de maior significância tem ordem ímpar.

\item Divisibilidade por $13$: Um número é divisível por $13$ se o quádruplo do último algarismo, somado ao número sem o último algarismo, for divisível por $13$.

\item Divisibilidade por $17$: Um número é divisível por $17$ quando o quíntuplo do último algarismo, subtraído do número que não contém este último algarismo, for divisível por $17$.

\item Divisibilidade por $19$: Um número é divisível por $19$ quando o dobro do último algarismo, somado ao número que não contém este último algarismo, for divisível por $19$.

\item Divisibilidade por $23$: Um número é divisível por $23$ quando o héptuplo do último algarismo, somado ao número que não contém este último algarismo, for divisível por $23$.

\item Divisibilidade por $29$: Um número é divisível por $29$ quando o triplo do último algarismo, subtraído do número que não contém este último algarismo, for divisível por $29$.

\item Divisibilidade por $31$: Um número é divisível por $31$ quando o triplo do último algarismo, somado ao número que não contém este último algarismo, for divisível por $31$.

\item Divisibilidade por $49$: Um número é divisível por $49$ quando o quíntuplo do último algarismo, somado ao número que não contém este último algarismo, for divisível por $49$.
\end{itemize}

\subsection*{Coisas Aleatórias}

\begin{itemize}
\item Um número inteiro $n$ pode ser escrito como a diferença de dois quadrados perfeitos $n = x^2-y^2$ se, e somente se, $n$ é ímpar ou múltiplo de $4$.

Se $n$ é multiplo de $4$: $n = 4k = (k+1)^2 - (k-1)^2$.

Se $n$ é ímpar: $n = 2k+1 = (k+1)^2 - k^2$.


\item Um número inteiro $n$ pode ser escrito como soma de $3$ quadrados se, e somente se, NÃO é da forma $n = 4^k*(8m+7)$.


\item O produto dos divisores positivos de $n$ é $\sqrt{n ^ {qtddDivisores}}$.

\end{itemize}



% \begin{figure}[!t]
% \centering
% \includegraphics[width=\textwidth]{leisenocosseno.png}
% \label{senocosseno}
% \end{figure}

% \begin{figure}[!t]
% \centering
% \includegraphics[width=\textwidth]{areatriangulo.jpg}
% \label{area}
% \end{figure}

% \begin{figure}[!t]
% \centering
% \includegraphics[width=\textwidth]{pontosnotaveis.png}
% \label{pontos}
% \end{figure}



\end{document}
