% Tipo do documento
\documentclass[12pt,a4paper, twoside]{report}

\usepackage{template}

\title{Material Geometria 2017}
\author{ % Jhúlia Graziella & Yuri Cardoso & Gustavo Nunes \\
%\vspace{1cm} \\
Faculdade de Computação \\
Universidade Federal de Uberlândia
}
\date{}

\begin{document}
  \maketitle
% -------------------------------------------------------------------- %
% Listas de figuras, tabelas e códigos criadas automaticamente
% \lstlistoflistings
% -------------------------------------------------------------------- %

% -------------------------------------------------------------------- %
% Sumário
\tableofcontents    

% Capítulos do trabalho

% cabeçalho para as páginas de todos os capítulos
\fancyhead[RE,LO]{\thesection}

%\singlespacing              % espaçamento simples
\setlength{\parskip}{0.15in} % espaçamento entre paragráfos

\chapter{Geometria}

\section{Área da união de retângulos $O(n log^2 n)$}
\noindent\begin{lstlisting}[caption=Área da união de retângulos (N log^2 N),language=C++]
struct seg{
    ll y, tipo;
    seg(){}
    seg(ll a, ll b){
        y = a;
        tipo = b;
    }
    bool operator < (seg other) const{
        return y > other.tipo;
    }
};

struct event{
    ll x, y, y0, tipo;
    event(){}
    event(ll a, ll b, ll c, ll d){
        x = a;
        y = b;
        y0 = c;
        tipo = d;
    }
    bool operator < (event other) const{
        return x < other.x;
    }
};

ll ABRE = 0;
ll FECHA = 1;

multiset<pair<ll, ll>, greater<pair<ll, ll> > > ativo;
vector<event> evento;

ll sweep(){
    ll area=0;
    ll x=0, x1=0;
    event eve;
    ll cnt = 0;
    ll U=0;
    ll D=0;
    //~ ll ans = 0;
    
    for (ll i = 0; i < evento.size(); i++)
    {
        eve = evento[i];
        cnt = 0;
        x1 = eve.x;
        //~ prllf("%s em x=%d\n", eve.tipo==ABRE ? "abre" : "fecha", eve.x);
        for(auto it : ativo){
            //~ oioi;
            if(it.S == ABRE){
                if(cnt==0) U = it.F;
                cnt++;
            }else if(it.S==FECHA){
                if(cnt==1){
                    D = it.F;
                    //~ prllf("x = %d  x1 = %d  U = %d  D = %d\n", x, x1, U, D);
                    area += (x1-x)*(U-D);
                }
                cnt--;
            }
        }
        
        if(eve.tipo == ABRE){
            ativo.insert(mp(eve.y0, ABRE));
            ativo.insert(mp(eve.y, FECHA));
        }else{
            //~ for(auto i : ativo){
                //~ if(i.F==eve.y && i.S==ABRE) ativo.erase(i);
                //~ if(i.F==eve.y0 && i.S==FECHA) ativo.erase(i);
            //~ }
            ativo.erase(ativo.find(mp(eve.y0, ABRE)));
            ativo.erase(ativo.find(mp(eve.y, FECHA)));
        }
        x = x1;
        //~ x1 = eve.x;
    }
    
    return area;
}

int main () {
    ll n, x, y, larg, alt, x0, y0;
    cin >> n;
    for (ll i = 0; i < n; i++)
    {
        cin >> x >> y >> x0 >> y0;
        evento.pb(event(x, y, y0, ABRE));
        evento.pb(event(x0, y, y0, FECHA));
    }
    sort(evento.begin(), evento.end());
    
    ll area = sweep();
    //~ cout << "Area total = " << area << endl;
    cout << area << endl;
    
    return 0;
}
\end{lstlisting}

\section{Área da união de retângulos $O(n log n)$}
\noindent\begin{lstlisting}[caption=Área da união de retângulos (N log N),language=C++]
#define P(x,y) make_pair(x,y)
using namespace std;
class Rectangle{
    public:
    int x1 , y1 , x2 , y2;
    static Rectangle empt;
    Rectangle(){
        x1=y1=x2=y2=0;
    }
    Rectangle(int X1 , int Y1 , int X2 , int Y2){
        x1 = X1; y1=Y1;
        x2 = X2; y2=Y2;
    }
    Rectangle intersect(Rectangle R){
        if(R.x1 >= x2 || R.x2 <= x1) return empt;
        if(R.y1 >= y2 || R.y2 <= y1) return empt;
        return Rectangle(max(x1 , R.x1) , max(y1 , R.y1) , min(x2 , R.x2) , min(y2 , R.y2));
    }
};
struct Event{
    int x , y1 , y2 , type;
    Event(){}
    Event(int x , int y1 , int y2 , int type):x(x) , y1(y1) , y2(y2) , type(type){}
};
bool operator < (const Event&A , const Event&B){
    //if(A.x != B.x)
    return A.x < B.x;
    //if(A.y1 != B.y1) return A.y1 < B.y1;
    //if(A.y2 != B.y2()) A.y2 < B.y2;
}
const int MX=(1<<17);
struct Node{
    int prob , sum , ans;
    Node(){}
    Node(int prob , int sum , int ans):prob(prob) , sum(sum) , ans(ans){}
};
Node tree[MX*4];
int interval[MX];
void build(int x , int a , int b){
    tree[x] = Node(0 , 0 , 0);
    if(a==b){
        tree[x].sum+=interval[a];
        return;
    }
    build(x*2 , a , (a+b)/2);
    build(x*2+1 , (a+b)/2+1 , b);
    tree[x].sum = tree[x*2].sum + tree[x*2+1].sum;
}
int ask(int x){
    if(tree[x].prob) return tree[x].sum;
    return tree[x].ans;
}
int st , en , V;
void update(int x , int a , int b){
    if(st>b || en<a) return;
    if(a>=st && b<=en){
        tree[x].prob+=V;
        return;
    }
    update(x*2 , a , (a+b)/2);
    update(x*2+1 , (a+b)/2+1 , b);
    tree[x].ans = ask(x*2) + ask(x*2+1);
}
Rectangle Rectangle::empt = Rectangle();
vector < Rectangle > Rect;
vector < int > sorted;
vector < Event > sweep;
void compressncalc(){
    sweep.clear();
    sorted.clear();
    for(auto R : Rect){
        sorted.push_back(R.y1);
        sorted.push_back(R.y2);
    }
    sort(sorted.begin() , sorted.end());
    sorted.erase(unique(sorted.begin() , sorted.end()) , sorted.end());
    int sz = sorted.size();
    for(int j=0;j<sorted.size() - 1;j++)
        interval[j+1] = sorted[j+1] - sorted[j];
    for(auto R : Rect){
        sweep.push_back(Event(R.x1 , R.y1 , R.y2 , 1));
        sweep.push_back(Event(R.x2 , R.y1 , R.y2 , -1));
    }
    sort(sweep.begin() , sweep.end());
    build(1,1,sz-1);
}
long long ans;
void Sweep(){
    ans=0;
    if(sorted.empty() || sweep.empty()) return;
    int last = 0 , sz_ = sorted.size();
    for(int j=0;j<sweep.size();j++){
        ans+= 1ll * (sweep[j].x - last) * ask(1);
        last = sweep[j].x;
        V = sweep[j].type;
        st = lower_bound(sorted.begin() , sorted.end() , sweep[j].y1) - sorted.begin() + 1;
        en = lower_bound(sorted.begin() , sorted.end() , sweep[j].y2) - sorted.begin();
        update(1 , 1 , sz_-1);
    }
}
int main(){
//    freopen("in.in","r",stdin);
    int n;
    scanf("%d",&n);
    for(int j=1;j<=n;j++){
        int a , b ,c  , d;
        scanf("%d %d %d %d",&a,&b,&c,&d);
        Rect.push_back(Rectangle(a , b , c , d));
    }
    compressncalc();
    Sweep();
    cout<<ans<<endl;
}

\end{lstlisting}

\section{Área da união de retângulos $O(n log n)$ - Outra implementação}
\noindent\begin{lstlisting}[caption=Área da união de retângulos (N log N) - Outra implementação,language=C++]
/** Union Of rectangle Area */ 

struct Edge {
    bool open;
    int x, yMin, yMax;
    Edge(int x, int y1, int y2, bool op) {
        this->x = x;
        yMin = y1, yMax = y2;
        open = op;
    }
    bool operator < (const Edge &e) const {
        return (x < e.x);
    }
};


int n, m, h[maxN << 1];
int sum[maxN << 5], counter[maxN << 5];
vector<Edge> edges;



void print ( )
{   int maxsize=ceil(log2(num));
     maxsize=2*pow(2,maxsize)-1;

   for(int i= 1 ; i<=maxsize ; i++ )
          cout<<sum[i]<<" ";
    
    cout<<endl;

}

void update(int p, int l, int r, int yMin, int yMax, bool open) {

  if (h[r] < yMin || yMax < h[l]) return;

  // if ymin is greater than h[r] which is array of sorted y coordinates or
  // ymax is less than h[l] then l - r is not the required range

 /*
suppose our figure is like this ::
          _(2,5)
        _|_|_ (3,4)
(0,3)  |_|_|_|
  (1,2) _|_|_ (3,2)
(0,1)  |_|_|_|
  Three rectangles are there
    1) 0,1 -> 3,2
    2) 0,3 -> 3,4
    3) 1,2 -> 2,5
  h   -> [ 1 , 2 , 3,  4, 5 ]
edges -> [ (0,1,2) (0,3,4) (1,2,5) (2,2,5) (3,1,2) (3,3,4) ]
       Segment Tree using updation
            (1,5)                        0         1          2          4          2         1         0
     (1,3)        (3,5)                 0 0       1 0        1 1        2 2        1 1       0 1       0 0
  (1,2) (2,3)  (3,4) (4,5)            0 0 0 0   1 0 0 0    1 0 1 0    1 1 1 0    1 0 1 0   0 0 1 0   0 0 0 0
Question is what is actually done over here ?
   1 ) Firstly we sorted the edges according to the x coordinate and h[] according to y coordinates.
   2 ) then we started moving horizontal using a vertical sweep line to encounter 2  types of events : Left Edge,  Right Edge
     i) Left Edge: on encountering a left edge, we move it into the active sets and update the total length
                   of vertical sweep line that is cut by the rectangualar boxes , this is done by update function
                   here we have two variable ymin (lower left y coordinate), ymax (upper left y coordinate) .
                   using which we update the total length of sweep line intersected at that event, and after updating
                   sum[1] gives the length of intersected sweep line at that x coordinate.
     ii) Right Edge : On encountering a right edge, we again update the segment tree.
   3) Function of segment tree is to store the intesected length at various x positions namely the events, whenever
      we reach an edge we update the intersected length  , whenever we are at left edge we know that we have added an edge between
      ymin and ymax therefore we need to add this to the sum at this position, if we are at right edge we know that we need to remove
      an edge between ymin and ymax, we do so by reducing counter ,if counter is still not zero means there were overlapping rectss
      and we sum[p] = h[r]- h[l] .
     NOTE : at any time there will be only two types of rectangles in active sets we need to worry abt
        ____                      __
       |____|        OR          |__|
           _____               __|__|__
          |_____|             |__|__|__|
  (One below or abv otr)     (overlapping once)
 */


    int c = p << 1, mid = (l + r) >> 1;

    if (yMin <= h[l] && h[r] <= yMax) {         // ymin --- h[l] --- h[r] --- ymax
        counter[p] += open ? 1 : -1;
        if (counter[p]) sum[p] = h[r] - h[l];  //if there is a rectangle at that posn that is bw h[l] and h[r] we will add that to length
        else sum[p] = sum[c] + sum[c + 1];    // else we will just sumup of lengths above and beloew this region
        return;
    }

    if (l + 1 >= r) return;

    update(c, l, mid, yMin, yMax, open);
    update(c + 1, mid, r, yMin, yMax, open);

    if (counter[p]) sum[p] = h[r] - h[l];
    else sum[p] = sum[c] + sum[c + 1];



}

int64 solve() {
                                // process height for horzntl. sweep line
    sort(h + 1, h + m + 1);    //  Sorting the hieght according to the y coordinates
    int k = 1;

    FOR (i, 2, m) if (h[i] != h[k])  //  Deleting the same horizontal sweeplines
       h[++k] = h[i];                //     as they are redundant
           m = k;
         num  = m;


    for (int i = 0, lm = maxN << 4; i < lm; i++) // This is the initialization step of segment tree
        sum[i] = 0, counter[i] = 0;



    int64 area = 0LL;  //  Initializing the Area

    sort(all(edges));  // Sorting according to x coordinates for ver. swp line

    update(1, 1, m, edges[0].yMin, edges[0].yMax, edges[0].open);
   // print();

    for (int i = 1; i < edges.size(); i++) {
        area += sum[1] * (int64)(edges[i].x - edges[i - 1].x);
        update(1, 1, m, edges[i].yMin, edges[i].yMax, edges[i].open);
      //  print();
            }
    return area;



}

int main() {
      int n;
      cin>>n;
      int  x1, y1, x2, y2;

        edges.clear();
        m = 0;
        FOR (i, 1, n) {
            scanf("%d %d %d %d", &x1, &y1, &x2, &y2);


            if(x1==x2 && y1>y2)swap(y1,y2);
            else if(y1==y2   && x1>x2)swap(x1,x2);

            x2+=1;    // x1 and y1 are bottom left coordinates
            y2+=1;    // x2 and y2 are top right coordinates


            edges.pb(Edge(x1, y1, y2, true));   // Inserting the Left edge
            edges.pb(Edge(x2, y1, y2, false));  // Inserting the Right Edge

            /*
            (x1,y2)     (x2,y2)
                _________
               |         |
   LeftEdge <- |         |-> Right Edge
               |_________|
            (x1,y1)     (x2,y1)
            */


            h[++m] = y1; // Inserting the Lower y Coordinate 1 based inddexiing
            h[++m] = y2; // Inserting the Upper y Coordinate

        }
        printf("%lld\n", solve());

    return 0;
}

\end{lstlisting}

\section{Estruturas}
\noindent\begin{lstlisting}[caption=Estruturas,language=C++]
struct pv{
    ld x, y;
    pv(){}
    pv(ld _x, ld _y){
        x = _x;
        y = _y;
    }
    
    int getQuad(){
        if(maiorIgual(x, 0.0) && maiorIgual(y, 0.0)) return 1;
        if(menor(x, 0.0) && maiorIgual(y, 0.0)) return 2;
        if(menorIgual(x, 0.0) && menor(y, 0.0)) return 3;
        return 4;
    }
    
    pv operator + (pv other){
        return pv(x+other.x, y+other.y);
    }
    
    pv operator - (pv other){
        return pv(x-other.x, y-other.y);
    }
    
    pv operator * (ld k){
        return pv(x*k, y*k);
    }
    
    pv operator / (ld k){
        return pv(x/k, y/k);
    }
    
    bool operator == (pv other){
        return igual(x, other.x) && igual(y, other.y);
    }
};

struct line{
    pv p0, v;
    line(){}
    line(pv _p0, pv _v){
        p0 = _p0;
        v = _v;
    }
};

struct seg{
    pv a, b;
    seg(){}
    seg(pv _a, pv _b){
        a = _a;
        b = _b;
    }
};

struct circle{
    pv centro;
    ld r;
    circle(){}
    circle(pv _centro, ld _r){
        centro = _centro;
        r = _r;
    }
};

\end{lstlisting}

\section{Funções básicas}
\noindent\begin{lstlisting}[caption=Funções básicas,language=C++]
bool igual(ld a, ld b){
    return fabs(a-b) < EPS;
}

bool maior(ld a, ld b){
    return a > EPS + b;
}

bool menor(ld a, ld b){
    return a + EPS < b;
}

bool maiorIgual(ld a, ld b){
    return maior(a, b) || igual(a, b);
}

bool menorIgual(ld a, ld b){
    return menor(a, b) || igual(a, b);
}

ld cross(pv a, pv b){
    return a.x * b.y - a.y * b.x;
}

ld dot(pv a, pv b){
    return a.x * b.x + a.y * b.y;
}

ld distPt(pv a, pv b){
    return hypot(a.x-b.x, a.y-b.y);
}

//norma de um vetor
ld norma(pv a){
    return sqrt(dot(a, a));
}

//retorna a menor distancia entre um ponto e uma reta
ld distPtReta(pv pt, line r){
    
    return fabs(cross(r.v, pt - r.p0)) / norma(r.v);
    
}

//retorna a menor distancia de um ponto a um segmento
ld distPtSeg(pv pt, seg r){
    if(maiorIgual(dot(pt - r.a, r.b - r.a), 0.0) && maiorIgual(dot(pt - r.b, r.a - r.b), 0.0))
        return distPtReta(pt, line(r.a, r.b-r.a));
    
    return min(distPt(pt, r.a), distPt(pt, r.b));
}


//angulo entre a reta horizontal com y = 0 e o vetor v
ld polarAngle(pv v){
    ld x = v.x;
    ld y = v.y;
    
    ld aux = atan2(y, x);
    
    if(menor(aux, 0.0)) aux += 2.0*PI;
    
    return aux;
}

//radiano para grau   
ld toDegree(ld rad){
    return rad * 180 / PI;
}

//angulo entre dois vetores (menor angulo)     
ld angle2Vec(pv v, pv u){
    ld ang1 = polarAngle(v);
    ld ang2 = polarAngle(u);
    ld ans = 0.0;
    
    ans = min(fabs(ang1 - ang2), fabs(2.0*PI+ang1 - ang2));
    ans = min(ans, fabs(2.0*PI+ang2 - ang1));

    return ans;
}

//checa se os pontos a, b, c estão em sentido anti horário  
bool ccw(pv a, pv b, pv c){
    ld aux = cross(b-a, c-b);
    
    return maior(aux, 0.0);
}

bool colinear(pv a, pv b, pv c){
    return igual(cross(b-a, c-b), 0.0);
}

//retorna o sinal de um número 
int sinal(ld res){
    if(maior(res, 0.0)) return 1;
    if(menor(res, 0.0)) return -1;
    return 0;
}

\end{lstlisting}

\section{Funções de comparação}
\noindent\begin{lstlisting}[caption=Funções de comparação,language=C++]
//ordena por x, se empatar por y     
bool compareConvexHull(pv a, pv b){
    if(igual(a.x, b.x)) return menor(a.y, b.y);
    return menor(a.x, b.x);
}

//função pra ordenar em relação ao angulo polar só com cross 
bool comparePolarAngle(pv a, pv b){
    if(a==b) return true;
    int q1 = a.getQuad();
    int q2 = b.getQuad();
    
    if(q1==q2){
        if(igual(cross(a, b), 0.0)) return menor(norma(a), norma(b));
        return maior(cross(a, b), 0.0);
    }
    return q1 < q2;
}
\end{lstlisting}

\section{Funções de reta}
\noindent\begin{lstlisting}[caption=Funções de reta,language=C++]
//checa se duas retas são paralelas       
bool retasParalelas(line r, line s){
    pv v = r.v;
    pv u = s.v;
    return igual(cross(v, u), 0.0);
}

//checa se duas retas são iguais      
bool retasIguais(line r, line s){
    pv p0 = r.p0;
    pv p1 = s.p0;
    pv v = r.v;
    pv u = s.v;
    
    return retasParalelas(r, s) && igual(cross(v, p1-p0), 0.0);
}

//retorna o ponto de interseccao de duas retas   
pv ptInterReta(line r, line s){
    pv v = r.v;
    pv u = s.v;
    
    pv p0 = r.p0;
    pv p1 = s.p0;
    
    long double t = (u.x * (p0.y - p1.y) + u.y * (p1.x - p0.x)) / cross(v, u);
    
    return pv(p0.x + (v.x * t), p0.y + (v.y * t));
}

//função que retorna a mediatriz 
line getMediatriz(pv a, pv b){
    pv vet, medio, perp;
    medio = (a+b)/2.0;
    vet = b-a;
    perp = pv(-vet.y, vet.x);
    
    return line(medio, perp);
}
\end{lstlisting}

\section{Funções de polígono}
\noindent\begin{lstlisting}[caption=Funções de polígono,language=C++]
//calcula o perimetro de um poligono       
ld perimetro(vector<pv> &polygon){
    if((int)polygon.size()<=1) return 0.0;
    
    if(!(polygon[0] == polygon[(int)polygon.size()-1]))
        polygon.pb(polygon[0]);
    
    ld ans = 0.0;
    for (int i = 0; i < (int)polygon.size()-1; i++)
    {
        ans += distPt(polygon[i], polygon[i+1]);
    }
    return ans;
}

//calcula a area de um poligono    
ld area(vector<pv> &polygon){
    if((int)polygon.size()<=2) return 0.0;
    
    if(!(polygon[0] == polygon[(int)polygon.size()-1]))
        polygon.pb(polygon[0]);
    
    ld ans = 0.0;
    for (int i = 0; i < (int)polygon.size()-1; i++)
    {
        ans += cross(polygon[i], polygon[i+1]);
    }
    return fabs(ans)*0.5;
}

//gera a convex hull do vector pt    
void convexHull(vector<pv> &polygon){
    if((int)polygon.size() <= 2) return;
    pv p1[(int)polygon.size() + 2];
    pv p2[(int)polygon.size() + 2];
    int sz1 = 0, sz2 = 0;
    
    sort(polygon.begin(), polygon.end(), compareConvexHull);
    for (int i = 0; i < (int)polygon.size(); i++)
    {
        while (sz1 > 1 && ccw(p1[sz1-2], p1[sz1-1], polygon[i]))
        {
            sz1--;
        }
        p1[sz1++] = polygon[i];
        
        while (sz2 > 1 && !ccw(p2[sz2-2], p2[sz2-1], polygon[i]))
        {
            sz2--;
        }
        p2[sz2++] = polygon[i];
    }
    
    polygon.clear();
    for (int i = 0; i < sz1; i++)
    {
        polygon.pb(p1[i]);
    }
    
    for (int i = sz2-2; i >= 1; i--)
    {
        polygon.pb(p2[i]);
    }
}

//checar se ponto esta dentro de poligono em O(n) - comparação de ângulo    
bool insidePolygon(pv p, vector<pv> &polygon){
    if((int)polygon.size()<= 2) return false;
    
    if(!(polygon[0] == polygon[(int)polygon.size()-1])){
        polygon.pb(polygon[0]);
    }
    bool naBorda = false;
    
    ld ang = 0.0;
    for (int i = 0; i < polygon.size()-1; i++)
    {
        naBorda = naBorda || onSegment(p, seg(polygon[i], polygon[i+1]));
        if(ccw(p, polygon[i], polygon[i+1])){
            ang += angle2Vec(polygon[i] - p, polygon[i+1]-p);
        }else{
            ang -= angle2Vec(polygon[i] - p, polygon[i+1]-p);
        }
    }
    return igual(fabs(ang), 2.0*PI) || naBorda;//se quiser totalmente dentro tem que ver se ele ta na borda
}

//checa se um ponto esta dentro de um triangulo (a, b, c)   
bool insideTriangle(pv p, pv a, pv b, pv c){
    if(onSegment(p, seg(a, b))) return true;
    if(onSegment(p, seg(a, c))) return true;
    if(onSegment(p, seg(b, c))) return true;
    
    return sinal(cross(b-a, p-a)) == sinal(cross(c-b, p-b)) && sinal(cross(c-b, p-b)) == sinal(cross(a-c, p-c));
}

//checa se um ponto esta dentro de um polígono em O(log n)   
bool insidePolygonLogN(pv p, vector<pv> &polygon){//NAO PODE TER PONTOS COLINEARES
    //CUIDADO COM POLIGONOS DEGENERADOS
    if((int)polygon.size() == 0) return false;
    
    int sz;
    if(polygon[0] == polygon[(int)polygon.size()-1]) sz = (int)polygon.size()-1;
    else sz = (int)polygon.size();
    
    if(sz <= 2) return false;
    
    if(ccw(polygon[0], polygon[1], polygon[2])){
        if(!(polygon[0] == polygon[(int)polygon.size()-1])) polygon.pb(polygon[0]);
        reverse(polygon.begin(), polygon.end());
    }
    
    int lo = 1, hi = sz-1, mid;
    int c;
    while (lo <= hi)
    {
        mid = (lo + hi)/2;
        c = sinal(cross(polygon[mid] - polygon[0], p - polygon[0]));
        if(c < 0){
            lo = mid+1;
        }else if(c > 0){
            hi = mid-1;
        }else{
            if(mid == 1){
                return insideTriangle(p, polygon[0], polygon[mid], polygon[mid+1]);
            }else{
                return insideTriangle(p, polygon[0], polygon[mid-1], polygon[mid]);
            }
        }
    }
    mid = (lo+hi)/2;
    if(mid == 0 || mid == sz-1) return false;
    
    return insideTriangle(p, polygon[0], polygon[mid], polygon[mid+1]);
}

bool insidePolygonLogNConfiavel(pv p, vector<pv>& polygon){
    vector<pv> lower, upper;
    convexHull(polygon, lower, upper);//PRECISA QUE A CONVEX HULL RETORNE A LOWER E UPPER HULL
    ld xMax, xMin, yMax, yMin;
    ld up1, up2, dw1, dw2;
    
    
    xMax = upper[(int)upper.size()-1].x;
    xMin = upper[0].x;
    
    for (int i = 0; i < (int)upper.size(); i++)
    {
        yMax = max(yMax, upper[i].y);
        yMin = min(yMin, upper[i].y);
        if(upper[i].x == xMin){
            up1 = max(up1, upper[i].y);
            dw1 = min(dw1, upper[i].y);
        }
        if(upper[i].x == xMax){
            up2 = max(up2, upper[i].y);
            dw2 = min(dw2, upper[i].y);
        }
    }
    for (int i = 0; i < (int)lower.size(); i++)
    {
        yMax = max(yMax, lower[i].y);
        yMin = min(yMin, lower[i].y);
        if(upper[i].x == xMin){
            up1 = max(up1, lower[i].y);
            dw1 = min(dw1, lower[i].y);
        }
        if(upper[i].x == xMax){
            up2 = max(up2, lower[i].y);
            dw2 = min(dw2, lower[i].y);
        }
    }
    if(p.x < xMin || p.x > xMax || p.y < yMin || p.y > yMax) return false;
    
    if(p.x == xMin && (p.y < dw1 || p.y > up1)) return false;
    if(p.x == xMax && (p.y < dw2 || p.y > up2)) return false;
    
    //nesse momento, o ponto tem x > xMin e x < xMax
    int lo = 0, hi = (int)upper.size()-1;
    int mid;
    int ans1, ans2;
    while (lo <= hi)
    {
        mid = (lo+hi)/2;
        if(p.x >= upper[mid].x && p.x <= upper[mid+1].x){
            ans1 = mid;
            break;
        }
        if(p.x < upper[mid].x){
            hi = mid-1;
        }else{
            lo = mid+1;
        }
    }
    
    lo = 0; hi = (int)lower.size()-1;
    while (lo <= hi)
    {
        mid = (lo+hi)/2;
        if(p.x >= lower[mid].x && p.x <= lower[mid+1].x){
            ans2 = mid;
            break;
        }
        if(p.x < lower[mid].x){
            hi = mid-1;
        }else{
            lo = mid+1;
        }
    }
    
    if(onSegment(p, seg(upper[ans1], upper[ans1+1]))) return true;
    if(onSegment(p, seg(lower[ans2], lower[ans2+1]))) return true;
    
    return sinal(cross(upper[ans1+1]-upper[ans1], p - upper[ans1]))==-1 && sinal(cross(lower[ans2+1] - lower[ans2], p - lower[ans2]))==1;
}

//FUNÇÃO DO JUNIOR (% significa cross)
bool PointIsInsideConvexPolygon(vector<pv> &a, pv p) {
  pv vp = p - a[0];
  if(cross((a[1] - a[0]), vp) > 0) return 0;
  int lo = 1, hi = a.size() - 1;
  while(lo < hi) {
    int md = (lo + hi + 1) >> 1;
    if(cross((a[md] - a[0]), vp) < 0) lo = md;
    else hi = md - 1;
  }
  if(hi == a.size() - 1) return false;
  return cross(((a[lo + 1] - a[lo]), (p - a[lo]))) < 0;
}

//retorna a interseccao dos poligonos 
int interPolygon(vector<pv> &polygon1, vector<pv>& polygon2, vector<pv> &interseccao){
    
    if(!(polygon1[0] == polygon1[(int)polygon1.size() - 1])) polygon1.pb(polygon1[0]);
    if(!(polygon2[0] == polygon2[(int)polygon2.size() - 1])) polygon2.pb(polygon2[0]);
    interseccao.clear();
    
    bool tem = false;
    vector<seg> segmentos;
    pv pt;
    
    for (int i = 0; i < (int)polygon1.size()-1; i++)
    {
        segmentos.pb(seg(polygon1[i], polygon1[i+1]));
    }
    for (int i = 0; i < (int)polygon2.size()-1; i++)
    {
        segmentos.pb(seg(polygon2[i], polygon2[i+1]));
    }
    
    int res;
    for (int i = 0; i < (int)segmentos.size(); i++)
    {
        for (int j = i+1; j < (int)segmentos.size(); j++)
        {
            res = interSegmento(segmentos[i], segmentos[j], pt);
            if(res == 1){
                if(insidePolygonLogN(pt, polygon1) && insidePolygonLogN(pt, polygon2)){
                    interseccao.pb(pt);
                    tem = true;
                }
            }else if(res == 2){
                if(onSegment(segmentos[i].a, segmentos[j])){
                    pt = segmentos[i].a;
                    if(insidePolygonLogN(pt, polygon1) && insidePolygonLogN(pt, polygon2)){
                        interseccao.pb(pt);
                        tem = true;
                    }
                }
                if(onSegment(segmentos[i].b, segmentos[j])){
                    pt = segmentos[i].b;
                    if(insidePolygonLogN(pt, polygon1) && insidePolygonLogN(pt, polygon2)){
                        interseccao.pb(pt);
                        tem = true;
                    }
                }
                if(onSegment(segmentos[j].a, segmentos[i])){
                    pt = segmentos[j].a;
                    if(insidePolygonLogN(pt, polygon1) && insidePolygonLogN(pt, polygon2)){
                        interseccao.pb(pt);
                        tem = true;
                    }
                }
                if(onSegment(segmentos[j].b, segmentos[i])){
                    pt = segmentos[j].b;
                    if(insidePolygonLogN(pt, polygon1) && insidePolygonLogN(pt, polygon2)){
                        interseccao.pb(pt);
                        tem = true;
                    }
                }
            }
        }
        
    }
    if(!tem) return false;
    
    convexHull(interseccao);
    
    return true;
}

//funcao usada na cutPolygon para não repetir vértices em um corte
void adiciona(vector<pv> &polygon, pv pt){
    if((int)polygon.size() == 0){
        polygon.pb(pt);
    }else{
        if(!(polygon[(int)polygon.size()-1] == pt))
            polygon.pb(pt);
    }
}

//NAO ACEITA PONTOS COLINEARES
//0 - não cortou ou cortou na borda
//1 - as duas partes tem area
int cutPolygon(line r, vector<pv> &polygon, vector<pv> &lp, vector<pv> &rp){
    if(!(polygon[0] == polygon[(int)polygon.size()-1])) 
        polygon.pb(polygon[0]);
    
    lp.clear();
    rp.clear();
    int lado;
    
    int cortou = 0;
    pv pt;
    for (int i = 0; i < (int)polygon.size()-1; i++)
    {
        lado = sinal(cross(r.v, polygon[i] - r.p0));
        
        //teste de ponto
        if(lado == 1){//esta à esquerda da linha
            adiciona(lp, polygon[i]);
        }else if(lado == -1){//esta à direita da linha
            adiciona(rp, polygon[i]);
        }
        
        //teste de intersecção
        if(sinal(cross(r.v, polygon[i] - r.p0)) != sinal(cross(r.v, polygon[i+1] - r.p0))){//segmento está inteiro na reta (cortou na borda)
            //nao faz nada pois os dois vertices serão adicionados na verificacao acima
            cortou = 1;
            pt = ptInterReta(line(polygon[i], polygon[i+1]-polygon[i]), r);
            adiciona(lp, pt);
            adiciona(rp, pt);
        }
    }
    if((int)lp.size() > 2) adiciona(lp, lp[0]);
    else lp.clear();
    if((int)rp.size() > 2) adiciona(rp, rp[0]);
    else rp.clear();
    return cortou;
}
\end{lstlisting}

\section{Funções de segmento}
\noindent\begin{lstlisting}[caption=Funções de segmento,language=C++]
//checa se ponto p está no segmento s   
bool onSegment(pv p, seg s){
    pv a = s.a;
    pv b = s.b;
    if(maiorIgual(dot(p-a, b-a), 0.0) && maiorIgual(dot(p-b, a-b), 0.0) && igual(cross(p-a, p-b), 0.0)) return true;
    return false;
}

//checa se 2 segmentos se cruzam (pode ser na borda)   
bool temInterSegmento(seg r, seg s){
    pv a = r.a;
    pv b = r.b;
    pv c = s.a;
    pv d = s.b;
    
    return sinal(cross(b-a, d-a)) != sinal(cross(b-a, c-a)) && sinal(cross(d-c, a-c)) != sinal(cross(d-c, b-c));
}

//checa se tem interseccao entre segmentos (0 - nao tem, 1 - tem, 2 - paralelos (SE TOCAM MAS NAO RETORNA O PONTO) )   
int interSegmento(seg r, seg s, pv &ans){
    pv a = r.a;
    pv b = r.b;
    pv c = s.a;
    pv d = s.b;
    if(retasIguais(line(a, b-a), line(c, d-c))){
        
        if(onSegment(c, r) || onSegment(d, r) || onSegment(a, s) || onSegment(b, s)){
            return 2;
        }
        return 0;
    }
    if(temInterSegmento(r, s)){
        ans = ptInterReta(line(a, b-a), line(c, d-c));
        return 1;
    }
    return 0;
}

\end{lstlisting}

\section{Line Sweep (Balão)}
\noindent\begin{lstlisting}[caption=Line Sweep (Balão),language=C++]
struct pv{
    ll x, y;
    pv(){}
    pv(ll _x, ll _y){
        x = _x;
        y = _y;
    }
    pv operator - (pv other){
        return pv(x-other.x, y-other.y);
    }
};

struct event{
    pv p;
    int tipo, id;
    event(){}
    event(pv _p, int _tipo, int _id){
        p = _p;
        tipo = _tipo;
        id = _id;
    }
    bool operator < (event other) const{
        if(p.x == other.p.x) return tipo < other.tipo;
        return p.x < other.p.x;
    }
};

struct seg{
    pv a, b;
    seg(){}
    seg(pv _a, pv _b){
        a = _a;
        b = _b;
    }
};


int dp[MAXN][LOG];
int qtdSeg, qtdBalao;
vector<seg> segmento;
vector<event> evento;
vector<pv> vertice;
vector<int> g[MAXN];

ll cross(pv a, pv b){
    return a.x*b.y - a.y*b.x;
}

ll func(pv a, pv b, pv c){
    return cross(b-a, c-b);
}

bool compare(int _1, int _2){
    pv a = segmento[_1].a;
    pv b = segmento[_1].b;
    pv c = segmento[_2].a;
    pv d = segmento[_2].b;
    
    if(a.x < c.x){
        if(func(a, c, b) > 0)
            return false;
        return true;
    }else{
        if(func(c, a, d) > 0)
            return true;
        return false;
    }
}
set<int, bool(*)(int, int)> s(compare);

void adicionar(int id){
    s.insert(id);
    set<int, bool(*)(int, int)>::iterator it;
    if(segmento[id].a.y > segmento[id].b.y){
        it = s.find(id);
        it++;
        if(it == s.end()){
            vertice[id] = pv(segmento[id].a.x, INF);
        }else{
            if(segmento[*it].a.y == segmento[*it].b.y){
                vertice[id] = pv(segmento[id].a.x, segmento[*it].a.y);
            }else{
                g[id].pb(*it);
            }
        }
    }
}

void remover(int id){
    set<int, bool(*)(int, int)>::iterator it;
    if(segmento[id].a.y < segmento[id].b.y){
        it = s.find(id);
        it++;
        if(it==s.end()){
            vertice[id] = pv(segmento[id].b.x, INF);
        }else{
            if(segmento[*it].a.y == segmento[*it].b.y){
                vertice[id] = pv(segmento[id].b.x, segmento[*it].a.y);
            }else{
                g[id].pb(*it);
            }
        }
    }
    s.erase(id);
}

void consultar(int id, int x){
    set<int, bool(*)(int, int)>::iterator it;
    if(s.size()==0){
        vertice[id] = pv(x, INF);
    }else{
        it = s.begin();
        if(segmento[*it].a.y == segmento[*it].b.y){
            vertice[id] = pv(x, segmento[*it].a.y);
        }else{
            g[id].pb(*it);
        }
    }
}

void sweep(){
    sort(evento.begin(), evento.end());
    for (int i = 0; i < evento.size(); i++)
    {
        if(evento[i].tipo == ABRE)
            adicionar(evento[i].id);
        else if(evento[i].tipo == FECHA)
            remover(evento[i].id);
        else
            consultar(evento[i].id, evento[i].p.x);
    }
}

void build(){
    for (int i = 0; i < qtdSeg + qtdBalao; i++)
    {
        if(g[i].size() == 0){
            dp[i][0] = i;
        }else{
            dp[i][0] = g[i][0];
        }
    }
    for (int j = 1; j < LOG; j++)
    {
        for (int i = 0; i < qtdSeg + qtdBalao; i++)
        {
            dp[i][j] = dp[dp[i][j-1]][j-1];
        }
    }
}

int pula(int u){
    int d = MAXN;
    for (int i = 0; i < LOG; i++)
    {
        if(d & (1<<i)){
            u = dp[u][i];
        }
    }
    return u;
}

void reset(){
    segmento.clear();
    evento.clear();
    vertice.clear();
    for (int i = 0; i < qtdSeg+qtdBalao; i++)
    {
        g[i].clear();
    }
}

int main(){
    ios_base::sync_with_stdio(0);
    cin.tie(0);
    
    while(cin >> qtdSeg >> qtdBalao){
        reset();
        ll x, y, X2, Y2;
        for (int i = 0; i < qtdSeg; i++)
        {
            cin >> x >> y >> X2 >> Y2;
            if(x < X2){
                segmento.pb(seg(pv(x, y), pv(X2, Y2)));
            }else{
                segmento.pb(seg(pv(X2, Y2), pv(x, y)));
            }
            evento.pb(event(segmento[i].a, ABRE, i));
            evento.pb(event(segmento[i].b, FECHA, i));
            vertice.pb(pv(-1, -1));
        }
        
        for (int i = 0 + qtdSeg; i < qtdBalao + qtdSeg; i++)
        {
            cin >> x;
            evento.pb(event(pv(x, 0), CONSULTA, i));
            vertice.pb(pv(-1, -1));
        }
        
        sweep();
        build();
        int ult;
        for (int i = 0 + qtdSeg; i < qtdBalao+qtdSeg; i++)
        {
            ult = pula(i);
            if(vertice[ult].y == INF) cout << vertice[ult].x << "\n";
            else cout << vertice[ult].x << " " << vertice[ult].y << "\n";
        }
    }
    return 0;
}
\end{lstlisting}

\section{Rotação de ponto}
\noindent\begin{lstlisting}[caption=Rotação de ponto,language=C++]
//angulo deve estar em radiano
pv rotacionaHorario(pv p, double rad)
{
    rad=-rad;
    return pv(p.x * cos(rad) - p.y * sin(rad), p.x * sin(rad) + p.y * cos(rad));
}

pv rotacionaAntiHorario(pv p, double rad)
{
    return pv(p.x * cos(rad) - p.y * sin(rad), p.x * sin(rad) + p.y * cos(rad));
}

int main()
{
    //le o ponto
    //chama a funcao pra rotacionar
    return 0;
}
\end{lstlisting}

\section{Rotating calipers}
\noindent\begin{lstlisting}[caption=Rotating calipers,language=C++]
int rotatingCalipers(vector<pv> &up, vector<pv> &dn){
    int ans = 0;
    
    int i = 0, j = dn.size()-1; 
    
    while(i < (int)up.size() - 1 || j > 0){
        // Entrou aqui: up[i] e dn[j] eh um antipodal pair
        ans = max(ans, distPt(up[i],dn[j]));//NAO TIRAR A RAIZ NO DISTPT PARA EVITAR PRECISAO
        
        if(i == (int)up.size()-1) j--;
        else if(j == 0) i++;
        else{
            // Verifica qual o menor angulo a ser rotacionado p utilizar na rotacao
            if((up[i+1].y - up[i].y) * (dn[j].x - dn[j-1].x)
                > (dn[j].y - dn[j-1].y) * (up[i+1].x - up[i].x ))
                i++;
            else
                j--;
        }
    }
    return ans;
}

\end{lstlisting}

\section{Sweep Circle (esconde-esconde)}
\noindent\begin{lstlisting}[caption=Sweep Circle (esconde-esconde),language=C++]
#define ABRE 0 
#define QUERY 1 
#define FECHA 2 

struct pv{
    ll x, y;
    pv(){}
    pv(ll _x, ll _y){
        x = _x;
        y = _y;
    }
    
    pv operator - (pv other){
        return pv(x-other.x, y-other.y);
    }
};

ll cross(pv a, pv b){
    return a.x * b.y - a.y * b.x;
}

struct event{
    pv pt;
    int tipo, quadrante, id;
    event(){}
    event(pv _pt, int _tipo, int _quadrante, int _id){
        pt = _pt;
        tipo = _tipo;
        quadrante = _quadrante;
        id = _id;
    }
    bool operator < (event other) const{
        ll c;
        if(quadrante == other.quadrante){
            c = cross(pt, other.pt);
            if(c == 0){
                return tipo < other.tipo;
            }
            return c > 0;
        }
        return quadrante < other.quadrante;
    }
};

struct seg{
    pv a, b;
    int id;
    seg(){}
    seg(pv _a, pv _b){
        a = _a;
        b = _b;
        id = -1;
    }
    seg(pv _a, pv _b, int _id){
        a = _a;
        b = _b;
        id = _id;
    }
};


ld distPt(pv a, pv b){
    return hypot(a.x-b.x, a.y-b.y);
}


ll dot(pv a, pv b){
    return a.x * b.x + a.y * b.y;
}

int sinal(ll res){
    if(res > 0) return 1;
    if(res < 0) return -1;
    return 0;
}

//checa se 2 segmentos se cruzam (pode ser na borda) 
bool temInterSegmento(seg r, seg s){
    pv a = r.a;
    pv b = r.b;
    pv c = s.a;
    pv d = s.b;
    
    return sinal(cross(b-a, d-a)) != sinal(cross(b-a, c-a)) && sinal(cross(d-c, a-c)) != sinal(cross(d-c, b-c));
}

void printaPonto(pv a){
    printf("(%lld, %lld)  ", a.x, a.y);
    
}

vector<pv> crianca;
vector<seg> parede, tmpParede;
vector<event> eve;
int qtdParede, qtdCrianca, qtdProcura;
pv origin;

bool compareSet(int id1, int id2){
    pv a = tmpParede[id1].a;
    pv b = tmpParede[id1].b;
    pv c = tmpParede[id2].a;
    pv d = tmpParede[id2].b;
    
    return cross(d-a, b-a) > 0;
}

//~ set<int, bool(*)(int, int)> active(compareSet);
set<int> active;

int getQuadrante(pv pt){
    ll x = pt.x;
    ll y = pt.y;
    if(x >= 0 && y >= 0) return 1;
    if(x < 0 && y >= 0) return 2;
    if(x <= 0 && y < 0) return 3;
    if(x > 0 && y < 0) return 4;
}

bool needCut(seg r){
    return r.a.y < 0 && r.b.y > 0;
}

int visivel(pv pt){
    //inicio bruta
    for (auto i : active)
    {
        seg r = tmpParede[i];
        if(temInterSegmento(seg(pv(0, 0), pt), r)) return 0;
    }
    return 1;
    
    //fim bruta
    if((int)active.size() == 0) return 1;
    seg r = tmpParede[*active.begin()];
    if(temInterSegmento(seg(pv(0, 0), pt), r)) return 0;
    return 1;
}

int sweep(){
    sort(eve.begin(), eve.end());
    
    int ans = 0;
    
    for (int i = 0; i < (int)eve.size(); i++)
    {
        if(eve[i].tipo == ABRE){
            active.insert(eve[i].id);
        }
        else if(eve[i].tipo == FECHA){
            active.erase(eve[i].id);
        }else ans += visivel(eve[i].pt);
    }
    return ans;
}

int solve(int idx){
    origin = crianca[idx];
    eve.clear();
    tmpParede.clear();
    pv pt;
    for (int i = 0; i < qtdCrianca; i++)
    {
        if(i==idx) continue;
        pt = crianca[i] - origin;
        eve.pb(event(pt, QUERY, getQuadrante(pt), -1));
    }
    
    int cnt = 0;
    seg nw;
    for (int i = 0; i < qtdParede; i++)
    {
        nw = seg(parede[i].a - origin, parede[i].b - origin, cnt);
        if(!needCut(nw)){
            if(cross(nw.a, nw.b) > 0){
                eve.pb(event(nw.a, ABRE, getQuadrante(nw.a), cnt));
                eve.pb(event(nw.b, FECHA, getQuadrante(nw.b), cnt));
            }else{
                eve.pb(event(nw.b, ABRE, getQuadrante(nw.b), cnt));
                eve.pb(event(nw.a, FECHA, getQuadrante(nw.a), cnt));
            }
            tmpParede.pb(nw);
            cnt++;
        }else{
            if(cross(nw.b - nw.a, pv(0, 0) - nw.a) > 0){
                eve.pb(event(pv(1,0), ABRE, 1, cnt));
                eve.pb(event(nw.b, FECHA, getQuadrante(nw.b), cnt));
                eve.pb(event(nw.a, ABRE, getQuadrante(nw.a), cnt));
                eve.pb(event(pv(1, 0), FECHA, 4, cnt));
            }else{
                eve.pb(event(nw.b, ABRE, getQuadrante(nw.b), cnt));
                eve.pb(event(pv(-1, 0), FECHA, 2, cnt));
                eve.pb(event(pv(-1, 0), ABRE, 3, cnt));
                eve.pb(event(nw.a, FECHA, getQuadrante(nw.a), cnt));
            }
            tmpParede.pb(nw);
            cnt++;
        }
    }
    return sweep();
}

void reset(){
    eve.clear();
    tmpParede.clear();
    parede.clear();
    crianca.clear();
}

int main(){
    ios_base::sync_with_stdio(0);
    cin.tie(0);
    
    ll x, y, x0, y0;
    while (cin >> qtdProcura >> qtdCrianca >> qtdParede)
    {
        reset();
        for (int i = 0; i < qtdCrianca; i++)
        {
            cin >> x >> y;
            crianca.pb(pv(x, y));
        }
        for (int i = 0; i < qtdParede; i++)
        {
            cin >> x >> y >> x0 >> y0;
            if(y < y0) parede.pb(seg(pv(x, y), pv(x0, y0), -1));
            else parede.pb(seg(pv(x0, y0), pv(x, y), -1));
        }
        for (int i = 0; i < qtdProcura; i++)
        {
            cout << solve(i) << "\n";
        }
    }
    
    
    return 0;
}

\end{lstlisting}

\section{União do volume das caixas}
\noindent\begin{lstlisting}[caption=União do volume das caixas,language=C++]
// OBS: nesse problema cada caixa era definido por um ponto (x,y,z) e pela origem (0,0,0)
#include <bits/stdc++.h>

using namespace std;

#define ft first
#define sd second
#define mp make_pair

typedef long long ll;

typedef struct ponto{
    ll x, y, z;
} Ponto;

typedef struct node{
    ll altMin;
    ll altMax;
    ll soma;
} Node;


bool cmp(Ponto a, Ponto b){
    if(a.x != b.x)
        return a.x > b.x;
    if(a.y != b.y)
        return a.y > b.y;
    return a.z > b.z;
}

Ponto v[100005];

// par de altura max / soma das alturas
Node st[400005];
ll lz[400005];

void build(int no, int l, int r){
    lz[no] = 0;
    st[no].altMin = st[no].altMax = st[no].soma = 0LL;

    if(l == r)
        return;

    build(no*2, l, (l+r)/2);
    build(no*2+1, 1+(l+r)/2, r);
    return;
}

void prop(int no, int l, int r){
    if(!lz[no]) 
        return;

    st[no].altMin = st[no].altMax = lz[no];
    st[no].soma = (r-l+1)*lz[no];
    if(l != r) 
        lz[no*2] = lz[no*2+1] = lz[no];

    lz[no] = 0;
    return;
}

int queryAlt(int no, int l, int r, int maior, int altura){ // me fala a primeira posicao <= maior q eh "mais baixa q o i" (tem z < vi.z)
    prop(no, l, r);
    if(st[no].altMax < altura)
        return l;
    if(l == r || st[no].altMin >= altura)
        return maior;
    return min(queryAlt(no*2, l, (l+r)/2, maior, altura), queryAlt(no*2+1, 1+(l+r)/2, r, maior, altura));
}

ll querySum(int no, int l, int r, int ini, int fim){
    prop(no, l, r);
    if(l > fim || r < ini)
        return 0LL;
    if(l >= ini && r <= fim)
        return st[no].soma;
    return querySum(no*2, l, (l+r)/2, ini, fim) + querySum(no*2+1, 1+(l+r)/2, r, ini, fim);
}

void update(int no, int l, int r, int ini, int fim, ll val){
    prop(no, l, r);
    if(l > fim || r < ini)
        return;
    if(l >= ini && r <= fim){
        lz[no] = val;
        prop(no, l, r);
        return;
    }

    update(no*2, l, (l+r)/2, ini, fim, val);
    update(no*2+1, 1+(l+r)/2, r, ini, fim, val);

    st[no].altMin = min(st[no*2].altMin, st[no*2+1].altMin);
    st[no].altMax = max(st[no*2].altMax, st[no*2+1].altMax);
    st[no].soma = st[no*2].soma + st[no*2+1].soma;
}

main(){
    while(1){
        ll n, m;
        scanf("%lld %lld", &n, &m);

        if(!n && !m) break;

        ll maxy = 0;
        for(int i=0;i<n;i++){
            scanf("%lld %lld %lld", &v[i].x, &v[i].y, &v[i].z);
            maxy = max(v[i].y, maxy);
        }
        maxy += 2;

        sort(v, v+n, cmp);
        build(1, 1, maxy);

        v[n].x = 0; // seta o menor x com zero
        ll ans = 0;
        for(int i=0;i<n;i++){
            int l = queryAlt(1, 1, maxy, maxy, v[i].z); // me fala a primeira posicao q eh "mais baixa q eu"
            if(l > v[i].y)
                continue;

            ll vaux = querySum(1, 1, maxy, l, v[i].y);
            ll vun = (v[i].y-l+1) * v[i].z - vaux;
            //printf("o volume de cada unidade eh %lld * %lld - %lld = %lld\n", v[i].y-l+1, v[i].z, vaux, vun);
            //printf("a qtdd de unidades eh %lld\n", v[i].x, v[i+1].x, v[i].x-v[i+1].x);
            //printf("o volume total do bloco eh %lld\n", vun * (v[i].x));
            //printf("vou dar update de %lld no range %d %lld\n\n", v[i].z, l, v[i].y);

            ans +=  vun * v[i].x;
            update(1, 1, maxy, l, v[i].y, v[i].z);
        }
        printf("%lld\n", m*m*m-ans);
    }
}
\end{lstlisting}



\end{document} 
